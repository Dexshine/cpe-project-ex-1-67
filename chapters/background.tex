\chapter{\ifenglish Background Knowledge and Theory\else ทฤษฎีที่เกี่ยวข้อง\fi}

การทำโครงงานนี้ ผู้จัดทำได้ศึกษาทฤษฎีและเทคโนโลยีที่เกี่ยวข้องกับการสร้างเว็บแอปพลิเคชัน
แบ่งได้เป็น 4 ส่วน ดังนี้
\section{ด้าน Frontend}
\subsection{HTML}
HTML คือ Hypertext Markup Language ซึ่งเป็นภาษาคอมพิวเตอร์ที่ใช้ในการสร้างและกำหนดโครงสร้างของเว็บไซต์ โดย HTML จะใช้ แท็ก (Tags) เพื่อจัดระเบียบเนื้อหา เช่น ข้อความ, รูปภาพ และลิงก์บนหน้าเว็บ 
\cite{html1}
\subsection{CSS}
CSS คือ Cascading Style Sheets ซึ่งเป็นภาษาที่ใช้ในการกำหนดรูปแบบและหน้าตาของเอกสารที่เขียนด้วยภาษา HTML หรือ XML โดย CSS ช่วยให้สามารถแยกเนื้อหาออกจากการออกแบบ ทำให้การจัดการและปรับเปลี่ยนรูปแบบของเว็บไซต์ทำได้ง่ายขึ้น
\cite{css1}\cite{css2}
\subsection{JavaScript}
JavaScript คือภาษาคอมพิวเตอร์ที่มีจุดเด่นในการสร้างการโต้ตอบและเพิ่มความน่าสนใจให้กับเว็บไซต์ 
โดยช่วยให้เว็บไซต์ไม่เพียงแค่แสดงผลข้อมูลแบบนิ่งๆ แต่สามารถเปลี่ยนแปลงข้อมูลบนเว็บไซต์ตามการตอบสนองของผู้ใช้ได้
\cite{javascript1}\cite{javascript2}
\subsection{React}
React คือ JavaScript Library ที่พัฒนาโดย Facebook (ปัจจุบันคือ Meta) ใช้สำหรับการสร้าง User Interfaces โดยเฉพาะอย่างยิ่งในรูปแบบของ components ซึ่งเป็นบล็อกของโค้ดที่สามารถนำกลับมาใช้ใหม่ได้
เพื่อช่วยให้การพัฒนาเว็บแอปพลิเคชันมีประสิทธิภาพและง่ายมากขึ้น \cite{react1}\cite{react2}
\subsection{Material UI}
Material UI คือ React component library ที่พัฒนาโดยชุมชน Open-source ซึ่งใช้หลักการออกแบบของ Google's Material Design เพื่อ
สร้างส่วนประกอบ UI ที่สวยงามและใช้งานง่ายสำหรับแอปพลิเคชัน React \cite{mui}\cite{mui2}
\section{ด้าน Backend}
\subsection{JSON}
JSON (JavaScript Object Notation) เป็นมาตรฐานในการแลกเปลี่ยนข้อมูลที่มีน้ำหนักเบาและเข้าใจง่าย ถูกพัฒนาขึ้นในช่วงต้นยุค 2000
JSON ได้รับความนิยมอย่างสูงในงานด้านการพัฒนา API และเว็บแอปพลิเคชัน เนื่องจากสามารถใช้งานได้ง่ายทั้งฝั่งเซิร์ฟเวอร์และไคลเอนต์ \cite{json}
\subsection{REST API}
REST API (Representational State Transfer Application Programming Interface) เป็นสถาปัตยกรรมที่ใช้ในการพัฒนาเว็บ API โดยมีหลักการ
และแนวทางที่ช่วยให้การสื่อสารระหว่างระบบต่างๆ เป็นไปอย่างมีประสิทธิภาพและยืดหยุ่น REST API ใช้ HTTP (Hypertext Transfer Protocol) ในการทำงาน ซึ่งช่วยให้สามารถเข้าถึงและจัดการ
ทรัพยากรต่างๆบนเซิร์ฟเวอร์ได้ \cite{restapi} 
\subsection{NodeJS}
NodeJS คือ สิ่งที่ทำหน้าที่เป็น JavaScript Runtime(ตัวแปรภาษา JavaScript ให้เป็นภาษาเครื่อง) โดย NodeJS จะทำให้ JavaScript ที่เคย run อยู่แค่ใน Browser เท่านั้น สามารถนำไป run
บน Server ได้ เมื่อภาษาที่ใช้ทั้งฝั่ง Server และ Client เหมือนกัน ทำให้การพัฒนาเว็บไซต์เป็นไปด้วยความรวดเร็ว \cite{node1}\cite{node2}
\subsection{ExpressJS}
ExpressJS คือ Framework สำหรับพัฒนาเว็บ ที่ทำงานบน NodeJS ซึ่งช่วยให้การสร้างแอปพลิเคชันและ API เป็นไปได้อย่างรวดเร็ว โดยมีคุณสมบัติเด่นหลายประการ เช่น ระบบการจัดการเส้นทาง (Routing), 
ฟังก์ชัน Middleware, การจัดการข้อผิดพลาด ทำให้ผู้พัฒนาสามารถสร้างแอปพลิเคชันที่มีความซับซ้อนสูงได้อย่างมีประสิทธิภาพ \cite{express1}\cite{express2}
\subsection{Mongoose}
Mongoose คือ Object Data Modeling (ODM) Library สำหรับ MongoDB ที่ทำงานบน NodeJS โดยช่วยในการจัดการความสัมพันธ์ระหว่างข้อมูล, การตรวจสอบโครงสร้างข้อมูล (schema validation) เป็นต้น \cite{mongoose}
\section{ด้าน Database}
\subsection{NoSQL}
NoSQL คือ แนวทางการออกแบบฐานข้อมูล ที่มุ่งเน้นการจัดเก็บและเรียกใช้ข้อมูลในรูปแบบที่ไม่ใช่ตาราง ซึ่งแตกต่างจากฐานข้อมูลเชิงสัมพันธ์ (Relational Database) ที่ใช้โครงสร้างตารางแบบดั้งเดิม
มักจะเหมาะสำหรับการจัดการข้อมูลขนาดใหญ่และไม่เป็นระเบียบ (Unstructured Data) เช่น ข้อมูลจากโซเชียลมีเดีย, IoT, และแอปพลิเคชันที่ต้องการการตอบสนองแบบเรียลไทม์ \cite{nosql}
\subsection{MongoDB}
์MongoDB คือฐานข้อมูลแบบ NoSQL ที่มีลักษณะการจัดเก็บข้อมูลในรูปแบบเอกสาร (Document-Oriented) โดยใช้โครงสร้างที่คล้ายกับ JSON ซ฿่งช่วยให้การจัดการข้อมูลมีความยืดหยุ่นและสามารถปรับขนาดได้ง่าย \cite{mongodb}

\section{ด้าน Authentication}
\subsection{JWT}
JWT (JSON Web Token) คือ มาตรฐานที่ใช้สำหรับการส่งข้อมูลระหว่างสองฝ่ายในรูปแบบ JSON โดยข้อมูลที่ส่งจะถูกตรวจสอบและเชื่อถือได้เนื่องจากมีการลงรายมือชื่อดิจิทัล (digital signature) ซึ่งสามารถทำได้โดยใช้กุญแจลับ (secret)
JWT ถูกออกแบบมาให้ใช้งานได้ในหลายบริบท โดยบทบาทเด่นคือ การตรวจสอบสิทธิ์ (authentication) และการมอบสิทธิ์ (authorization) \cite{jwt1}\cite{jwt2}
\subsection{OAuth 2.0}
OAuth 2.0 คือ Authorization Framework ที่ออกแบบมาเพื่อให้ผู้ใช้สามารถแชร์ข้อมูลของตนเองระหว่างแอปพลิเคชันต่างๆ ได้อย่างปลอดภัยโดยไม่ต้องเปิดเผยรหัสผ่านของตนเองให้กับ third-party applications \cite{oauth1}\cite{oauth2}

% \subsubsection{Subsubsection 1 heading goes here}
% Subsubsection 1 text

% \section{Third section}
% Section 3 text. The dielectric constant\index{dielectric constant}
% at the air-metal interface determines
% the resonance shift\index{resonance shift} as absorption or capture occurs
% is shown in Equation~\eqref{eq:dielectric}:

% \begin{equation}\label{eq:dielectric}
% k_1=\frac{\omega}{c({1/\varepsilon_m + 1/\varepsilon_i})^{1/2}}=k_2=\frac{\omega
% \sin(\theta)\varepsilon_\mathit{air}^{1/2}}{c}
% \end{equation}

% \noindent
% where $\omega$ is the frequency of the plasmon, $c$ is the speed of
% light, $\varepsilon_m$ is the dielectric constant of the metal,
% $\varepsilon_i$ is the dielectric constant of neighboring insulator,
% and $\varepsilon_\mathit{air}$ is the dielectric constant of air.

% \section{About using figures in your report}

% define a command that produces some filler text, the lorem ipsum.
\newcommand{\loremipsum}{
  \textit{Lorem ipsum dolor sit amet, consectetur adipisicing elit, sed do
  eiusmod tempor incididunt ut labore et dolore magna aliqua. Ut enim ad
  minim veniam, quis nostrud exercitation ullamco laboris nisi ut
  aliquip ex ea commodo consequat. Duis aute irure dolor in
  reprehenderit in voluptate velit esse cillum dolore eu fugiat nulla
  pariatur. Excepteur sint occaecat cupidatat non proident, sunt in
  culpa qui officia deserunt mollit anim id est laborum.}\par}

% \begin{figure}
%   \centering

%   \fbox{
%      \parbox{.6\textwidth}{\loremipsum}
%   }

%   % To include an image in the figure, say myimage.pdf, you could use
%   % the following code. Look up the documentation for the package
%   % graphicx for more information.
%   % \includegraphics[width=\textwidth]{myimage}

%   \caption[Sample figure]{This figure is a sample containing \gls{lorem ipsum},
%   showing you how you can include figures and glossary in your report.
%   You can specify a shorter caption that will appear in the List of Figures.}
%   \label{fig:sample-figure}
% \end{figure}

% Using \verb.\label. and \verb.\ref. commands allows us to refer to
% figures easily. If we can refer to Figures
% \ref{fig:walrus} and \ref{fig:sample-figure} by name in the {\LaTeX}
% source code, then we will not need to update the code that refers to it
% even if the placement or ordering of the figures changes.


% This code demonstrates how to get a landscape table or figure. It
% uses the package lscape to turn everything but the page number into
% landscape orientation. Everything should be included within an
% \afterpage{ .... } to avoid causing a page break too early.
% \afterpage{
%   \begin{landscape}
%   \begin{table}
%     \caption{Sample landscape table}
%     \label{tab:sample-table}

%     \centering

%     \begin{tabular}{c||c|c}
%         Year & A & B \\
%         \hline\hline
%         1989 & 12 & 23 \\
%         1990 & 4 & 9 \\
%         1991 & 3 & 6 \\
%     \end{tabular}
%   \end{table}
%   \end{landscape}
% }

% \loremipsum\loremipsum\loremipsum

% \section{Overfull hbox}

% When the \verb.semifinal. option is passed to the \verb.cpecmu. document class,
% any line that is longer than the line width, i.e., an overfull hbox, will be
% highlighted with a black solid rule:
% \begin{center}
% \begin{minipage}{2em}
% juxtaposition
% \end{minipage}
% \end{center}

\section{\ifenglish%
\ifcpe CPE \else ISNE \fi knowledge used, applied, or integrated in this project
\else%
ความรู้ตามหลักสูตรซึ่งถูกนำมาใช้หรือบูรณาการในโครงงาน
\fi
}

\begin{enumerate}
  \item ความรู้ด้าน Web development จากวิชา 261207 Basic Computer Engineering Laboratory
  \item ความรู้ด้าน ฐานข้อมูล จากวิชา 261342 Fundamentals of Database Systems
  \item ความรู้ด้าน Deployment จากวิชา 261497 Full Stack Development
\end{enumerate}

\section{\ifenglish%
Extracurricular knowledge used, applied, or integrated in this project
\else%
ความรู้นอกหลักสูตรซึ่งถูกนำมาใช้หรือบูรณาการในโครงงาน
\fi
}

\begin{enumerate}
  \item ความรู้ด้าน NoSQL และ MongoDB เพื่อใช้สร้างฐานข้อมูล
  \item ความรู้เกี่ยวกับบริการ Google Firebase เพื่อใช้สร้างพื้นที่จัดเก็บรูปภาพ
\end{enumerate}
