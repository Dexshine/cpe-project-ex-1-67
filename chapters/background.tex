\chapter{\ifenglish Background Knowledge and Theory\else ทฤษฎีที่เกี่ยวข้อง\fi}

การทำโครงงานนี้ ผู้จัดทำได้ศึกษาทฤษฎีและเทคโนโลยีที่เกี่ยวข้องกับการสร้างเว็บแอปพลิเคชัน
แบ่งได้เป็น 3 ส่วน
\section{ด้าน Frontend}
\subsection{HTML}
HTML คือ Hypertext Markup Language ซึ่งเป็นภาษาคอมพิวเตอร์ที่ใช้ในการสร้างและกำหนดโครงสร้างของเว็บไซต์ โดย HTML จะใช้ แท็ก (Tags) เพื่อจัดระเบียบเนื้อหา เช่น ข้อความ, รูปภาพ และลิงก์บนหน้าเว็บ 
\cite{html1}
\subsection{CSS}
CSS คือ Cascading Style Sheets ซึ่งเป็นภาษาที่ใช้ในการกำหนดรูปแบบและหน้าตาของเอกสารที่เขียนด้วยภาษา HTML หรือ XML โดย CSS ช่วยให้สามารถแยกเนื้อหาออกจากการออกแบบ ทำให้การจัดการและปรับเปลี่ยนรูปแบบของเว็บไซต์ทำได้ง่ายขึ้น
\cite{css1}\cite{css2}
\subsection{JavaScript}
JavaScript คือภาษาคอมพิวเตอร์ที่มีจุดเด่นในการสร้างการโต้ตอบและเพิ่มความน่าสนใจให้กับเว็บไซต์ 
โดยช่วยให้เว็บไซต์ไม่เพียงแค่แสดงผลข้อมูลแบบนิ่งๆ แต่สามารถเปลี่ยนแปลงข้อมูลบนเว็บไซต์ตามการตอบสนองของผู้ใช้ได้
\cite{javascript1}\cite{javascript2}
\subsection{React}
React คือ JavaScript Library ที่พัฒนาโดย Facebook (ปัจจุบันคือ Meta) ใช้สำหรับการสร้าง User Interfaces โดยเฉพาะอย่างยิ่งในรูปแบบของ components ซึ่งเป็นบล็อกของโค้ดที่สามารถนำกลับมาใช้ใหม่ได้
เพื่อช่วยให้การพัฒนาเว็บแอปพลิเคชันมีประสิทธิภาพและง่ายมากขึ้น\cite{react1}\cite{react2}
\subsection{Material UI}
Material UI คือ React component library ที่พัฒนาโดยชุมชน Open-source ซึ่งใช้หลักการออกแบบของ Google's Material Design เพื่อ
สร้างส่วนประกอบ UI ที่สวยงามและใช้งานง่ายสำหรับแอปพลิเคชัน React\cite{mui}\cite{mui2}
\section{ด้าน Backend}
\subsection{JSON}
\subsection{NodeJS}
\subsection{ExpressJS}
\subsection{Mongoose}

\section{ด้าน Database}
\subsection{NoSQL}
\subsection{MongoDB}

\section{ด้าน Authentication}
\subsection{JWT Token}
\subsection{OAuth 2.0}
Subsection 1 text

\subsubsection{Subsubsection 1 heading goes here}
Subsubsection 1 text

\subsubsection{Subsubsection 2 heading goes here}
Subsubsection 2 text

\section{Third section}
Section 3 text. The dielectric constant\index{dielectric constant}
at the air-metal interface determines
the resonance shift\index{resonance shift} as absorption or capture occurs
is shown in Equation~\eqref{eq:dielectric}:

\begin{equation}\label{eq:dielectric}
k_1=\frac{\omega}{c({1/\varepsilon_m + 1/\varepsilon_i})^{1/2}}=k_2=\frac{\omega
\sin(\theta)\varepsilon_\mathit{air}^{1/2}}{c}
\end{equation}

\noindent
where $\omega$ is the frequency of the plasmon, $c$ is the speed of
light, $\varepsilon_m$ is the dielectric constant of the metal,
$\varepsilon_i$ is the dielectric constant of neighboring insulator,
and $\varepsilon_\mathit{air}$ is the dielectric constant of air.

\section{About using figures in your report}

% define a command that produces some filler text, the lorem ipsum.
\newcommand{\loremipsum}{
  \textit{Lorem ipsum dolor sit amet, consectetur adipisicing elit, sed do
  eiusmod tempor incididunt ut labore et dolore magna aliqua. Ut enim ad
  minim veniam, quis nostrud exercitation ullamco laboris nisi ut
  aliquip ex ea commodo consequat. Duis aute irure dolor in
  reprehenderit in voluptate velit esse cillum dolore eu fugiat nulla
  pariatur. Excepteur sint occaecat cupidatat non proident, sunt in
  culpa qui officia deserunt mollit anim id est laborum.}\par}

\begin{figure}
  \centering

  \fbox{
     \parbox{.6\textwidth}{\loremipsum}
  }

  % To include an image in the figure, say myimage.pdf, you could use
  % the following code. Look up the documentation for the package
  % graphicx for more information.
  % \includegraphics[width=\textwidth]{myimage}

  \caption[Sample figure]{This figure is a sample containing \gls{lorem ipsum},
  showing you how you can include figures and glossary in your report.
  You can specify a shorter caption that will appear in the List of Figures.}
  \label{fig:sample-figure}
\end{figure}

Using \verb.\label. and \verb.\ref. commands allows us to refer to
figures easily. If we can refer to Figures
\ref{fig:walrus} and \ref{fig:sample-figure} by name in the {\LaTeX}
source code, then we will not need to update the code that refers to it
even if the placement or ordering of the figures changes.


% This code demonstrates how to get a landscape table or figure. It
% uses the package lscape to turn everything but the page number into
% landscape orientation. Everything should be included within an
% \afterpage{ .... } to avoid causing a page break too early.
\afterpage{
  \begin{landscape}
  \begin{table}
    \caption{Sample landscape table}
    \label{tab:sample-table}

    \centering

    \begin{tabular}{c||c|c}
        Year & A & B \\
        \hline\hline
        1989 & 12 & 23 \\
        1990 & 4 & 9 \\
        1991 & 3 & 6 \\
    \end{tabular}
  \end{table}
  \end{landscape}
}

\loremipsum\loremipsum\loremipsum

\section{Overfull hbox}

When the \verb.semifinal. option is passed to the \verb.cpecmu. document class,
any line that is longer than the line width, i.e., an overfull hbox, will be
highlighted with a black solid rule:
\begin{center}
\begin{minipage}{2em}
juxtaposition
\end{minipage}
\end{center}

\section{\ifenglish%
\ifcpe CPE \else ISNE \fi knowledge used, applied, or integrated in this project
\else%
ความรู้ตามหลักสูตรซึ่งถูกนำมาใช้หรือบูรณาการในโครงงาน
\fi
}

\begin{enumerate}
  \item ความรู้ด้าน Web development จากวิชา 261207
  \item ความรู้ด้าน ฐานข้อมูล จากวิชา 261342
  \item ความรู้ด้าน Deployment จากวิชา 261497
\end{enumerate}

\section{\ifenglish%
Extracurricular knowledge used, applied, or integrated in this project
\else%
ความรู้นอกหลักสูตรซึ่งถูกนำมาใช้หรือบูรณาการในโครงงาน
\fi
}

\begin{enumerate}
  \item ความรู้ด้าน NoSQL และ MongoDB เพื่อใช้สร้างฐานข้อมูล
  \item ความรู้เกี่ยวกับบริการ Google Firebase เพื่อใช้สร้างพื้นที่จัดเก็บรูปภาพ
\end{enumerate}
