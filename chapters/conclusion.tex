\chapter{\ifenglish Conclusions and Discussions\else บทสรุปและข้อเสนอแนะ\fi}

\section{\ifenglish Conclusions\else สรุปผล\fi}

จากผลการดำเนินงาน สามารถบรรลุวัตถุประสงค์ของโครงงานทั้ง 2 ประการคือ 1. สามารถประกาศกิจกรรมและแสดงความสนใจเข้าร่วมกิจกรรมได้ และ
2. สามารถรีวิวกิจกรรมที่สิ้นสุดแล้วได้ แต่ยังมีข้อจำกัดคือ การ Deploy แบบฟรี ทำให้ไม่สามารถรองรับจำนวนผู้ใช้เยอะๆ ได้

\section{\ifenglish Challenges\else ปัญหาที่พบและแนวทางการแก้ไข\fi}

\begin{enumerate}
    \item Frontend
    \begin{itemize}
        \item มีการเปลี่ยนแปลง UI/UX หลายครั้ง ทำให้การพัฒนาเกิดความล่าช้า แนวทางแก้ไขปัญหา คือ ควรวางแผนในขั้นตอนการออกแบบให้รัดกุมมากขึ้น
        \item การแก้ Bug ในโค้ดเป็นไปด้วยความล่าช้า เพราะภาษา JavaScript จะไม่แสดง error จนกว่าจะเริ่ม run โปรแกรม แนวทางแก้ไขปัญหา คือ ควรเปลี่ยนไปใช้ภาษา TypeScript
    \end{itemize}
    \item Backend
    \begin{itemize}
        \item ปัญหาการอัพโหลดรูปภาพกิจกรรมทำได้ช้าเกินไป เพราะการ deploy ฝั่งเซิร์ฟเวอร์ได้ใช้บริการแบบ Free ของ Heroku ซึ่งมีประสิทธิภาพไม่ดีเท่าที่ควร แนวทางแก้ไขปัญหา คือ ควรเปลี่ยนไปใช้บริการของบริษัทอื่น หรือใช้บริการแบบเสียเงิน
        \item ปัญหาการ sorting กิจกรรมสำหรับคุณ ทำได้ช้ากว่าที่ควร เนื่องจากตอนออกแบบฐานข้อมูล ยังไม่ได้คิดถึงเรื่องระบบแนะนำกิจกรรม ทำให้ออกแบบฐานข้อมูลและเลือกใช้ algorithm ที่ไม่มีประสิทธิภาพ แนวทางแก้ไขปัญหา คือ ควรทำขั้นตอน requirement ของระบบให้ชัดเจน เพื่อให้ไม่มีการเพิ่มฟีเจอร์ในภายหลัง
    \end{itemize}
\end{enumerate}

\section{\ifenglish%
Suggestions and further improvements
\else%
ข้อเสนอแนะและแนวทางการพัฒนาต่อ
\fi
}

\begin{enumerate}
    \item ชื่อ Collection ที่สื่อความหมายได้ไม่ชัดเจน
    \item Algorithm แนะนำแท็คยังไม่มีประสิทธิภาพ (O(n))
    \item Algorithm คิดคะแนนเฉลี่ยกิจกรรมยังไม่มีประสิทธิภาพ (O(n))
    \item บริการ Deploy แบบฟรีของบางบริษัทไม่มีประสิทธิภาพ
    \item ควรมีระบบแจ้งเตือนผู้ใช้ เมื่อกิจกรรมที่กดสนใจใกล้มาถึง
    \item ในเมื่อมีการเลือกคณะที่สามารถเข้าร่วมได้ ก็ควรมีการเลือกภาควิชา หรือสาขาวิชาด้วย
    \item การเพิ่ม/ลดเกณฑ์คะแนนกิจกรรมทำได้ยาก
\end{enumerate}
