\chapter{\ifenglish Introduction\else บทนำ\fi}

\section{\ifenglish Project rationale\else ที่มาของโครงงาน\fi}
\hspace{\parindent}บุคลากรภายในมหาวิทยาลัยเชียงใหม่จำนวนมากต้องการหาคนมาเข้าร่วมกิจกรรมต่างๆ แต่ผู้เข้าร่วมมีจำนวนน้อยหรืออาจไม่มีเลย
ซึ่งหนึ่งในเหตุผลที่ทำให้เกิดเหตุการณ์ดังกล่าวขึ้น คือ การประกาศกิจกรรมนั้นทำผ่านหลายแพลตฟอร์ม และในหนึ่งแพลตฟอร์มก็ยังแบ่งเป็นหลากหลายหน้า
ทำให้นักศึกษาและบุคลากรไม่ได้รับข่าวสารกิจกรรมทั้งหมดในมหาวิทยาลัยเชียงใหม่ 

ผู้จัดทำจึงพัฒนาเว็บแอปพลิเคชันที่เป็นศูนย์กลางให้นักศึกษาและบุคลากรของมหาวิทยาลัยเชียงใหม่สามารถประกาศกิจกรรมที่กำลังจะเกิดขึ้น 
แสดงความสนใจในการเข้าร่วมกิจกรรม และแบ่งปันรีวิวของกิจกรรมที่ผ่านมาได้
\section{\ifenglish Objectives\else วัตถุประสงค์ของโครงงาน\fi}
\begin{enumerate}
    \item พัฒนาเว็บแอปพลิเคชันที่นักศึกษาและบุคลากรสามารถประกาศกิจกรรมและแสดงความสนใจเข้าร่วมกิจกรรมได้
    \item มีระบบที่ผู้ใช้สามารถให้คะแนนกิจกรรมที่สิ้นสุดแล้วได้
\end{enumerate}

\section{\ifenglish Project scope\else ขอบเขตของโครงงาน\fi}

\subsection{\ifenglish Hardware scope\else ขอบเขตด้านฮาร์ดแวร์\fi}
คอมพิวเตอร์ส่วนบุคคลและแล็ปท็อป
\subsection{\ifenglish Software scope\else ขอบเขตด้านซอฟต์แวร์\fi}
\begin{enumerate}
    \item ระบบปฏิบัติการบนคอมพิวเตอร์ส่วนบุคคลที่สามารถใช้เว็บบราวเซอร์ได้ เช่น Microsoft Windows, Linux, MacOS เป็นต้น
    \item เว็บบราวเซอร์ต่างๆ เช่น Google Chrome, Mozilla Firefox, Microsoft Edge เป็นต้น
    \item มีการเข้าสู่ระบบโดยใช้ Google Account และ CMU Account
    \item มีระบบการแสดงกิจกรรมทั้งหมด
    \item มีระบบการประกาศกิจกรรม
    \item มีระบบการแสดงความสนใจเข้าร่วมกิจกรรม
    \item มีระบบรีวิวกิจกรรม
    \item มีแดชบอร์ดแสดงสถิติของเว็บ
\end{enumerate}
\subsection{\ifenglish Users scope\else ขอบเขตด้านกลุ่มผู้ใช้\fi}
นักศึกษาและบุคลากรของมหาวิทยาลัยเชียงใหม่
\subsection{\ifenglish Data scope\else ขอบเขตด้านข้อมูล\fi}
กิจกรรมประเภทต่างๆ เช่น จิตอาสา, การแสดงโชว์, การอบรมความรู้ เป็นต้น 
\section{\ifenglish Expected outcomes\else ประโยชน์ที่ได้รับ\fi}
\begin{enumerate}
    \item ทำให้การค้นหาข้อมูลกิจกรรมต่างๆในมหาวิทยาลัยเชียงใหม่สะดวกมากยิ่งขึ้น
    \item ทำให้นักศึกษาและบุคลากรในมหาวิทยาลัยเชียงใหม่ได้คนมาเข้าร่วมกิจกรรมของตนเองมากขึ้น
    \item ทำให้นักศึกษาและบุคลากรในมหาวิทยาลัยเชียงใหม่เห็นกิจกรรมตามความสนใจของผู้ใช้รายนั้น
\end{enumerate}
\section{\ifenglish Technology and tools\else เทคโนโลยีและเครื่องมือที่ใช้\fi}

\subsection{\ifenglish Hardware technology\else เทคโนโลยีด้านฮาร์ดแวร์\fi}
ASUS Vivobook Pro 15 : สำหรับพัฒนาเว็บแอปพลิเคชัน
\subsection{\ifenglish Software technology\else เทคโนโลยีด้านซอฟต์แวร์\fi}
\begin{enumerate}
    \item Figma : ใช้สำหรับออกแบบ UI/UX ของเว็บแอปพลิเคชัน
    \item Github : ใช้สำหรับการควบคุมเวอร์ชันของโค้ดและการอัพโหลดโค้ดออนไลน์
    \item Visual Studio Code : Text Editor ใช้สำหรับเขียนโค้ด
    \item Javascript : ภาษาโปรแกรมมิ่งสำหรับทั้ง frontend และ backend
    \item React : Javascript Library ที่ใช้สำหรับการสร้าง UI ที่มีประสิทธิภาพและรวดเร็ว
    \item NodeJS : เป็น Javascript Runtime ที่ใช้สำหรับทำ Server-side Javascript
    \item MongoDB : ฐานข้อมูล NoSQL ใช้สำหรับเก็บข้อมูลของเว็บ
    \item MongoDB Compass : เครื่องมือ GUI สำหรับจัดการและวิเคราะห์ข้อมูลใน MongoDB
    \item Mongoose : Object Data Modeling Library ใช้กำหนดโครงสร้างข้อมูลและการทำ queries 
    \item ExpressJS : ใช้ในการสร้าง RESTful API สำหรับ backend
    \item Firebase : ใช้ในการเก็บรูปภาพกิจกรรม
    \item Google Cloud : ใช้ในการสร้าง Google Account Authentication
    \item MaterialUI : React Component Library ใช้สำหรับสร้าง UI ที่สวยงาม
    \item Netlify, Heroku, MongoDB Atlas : ใช้สำหรับ hosting และ deployment ของ frontend, backend และ database ตามลำดับ
    \item Postman : ใช้สำหรับทดสอบการใช้งาน API
\end{enumerate}
\section{\ifenglish Project plan\else แผนการดำเนินงาน\fi}

\begin{plan}{6}{2024}{10}{2024}
    \planitem{6}{2024}{7}{2024}{ออกแบบฐานข้อมูล}
    \planitem{6}{2024}{9}{2024}{พัฒนาเว็บแอปพลิเคชัน}
    \planitem{9}{2024}{10}{2024}{ทดสอบและปรับปรุงระบบ}
    \planitem{10}{2024}{10}{2024}{เขียนรายงานและนำเสนอ}
\end{plan}

\section{\ifenglish Roles and responsibilities\else บทบาทและความรับผิดชอบ\fi}
นายณัฏฐพล ตันจอ 620610786 ทำหน้าที่ออกแบบฐานข้อมูล, ออกแบบ API และพัฒนาเว็บแอปพลิเคชัน
\section{\ifenglish%
Impacts of this project on society, health, safety, legal, and cultural issues
\else%
ผลกระทบด้านสังคม สุขภาพ ความปลอดภัย กฎหมาย และวัฒนธรรม
\fi}

เว็บแอปพลิเคชันนี้มีศักยภาพที่จะสร้างผลกระทบเชิงบวกต่อสังคมในหลายด้าน เช่น 
\begin{itemize}
    \item ส่งเสริมการมีส่วนร่วมในกิจกรรมต่างๆ ทำให้เกิดความรู้สึกเป็นส่วนหนึ่งของสังคมมากขึ้น
    \item เพิ่มโอกาสในการพัฒนาทักษะการสื่อสาร การทำงานเป็นทีม และภาวะผู้นำ
    \item กระตุ้นให้เกิดการเรียนรู้นอกห้องเรียนผ่านกิจกรรมที่หลากหลาย
\end{itemize}
